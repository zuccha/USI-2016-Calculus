% Amedeo Zucchetti
% May 1, 2016

\documentclass{article}

\usepackage[lmargin=2.5cm, rmargin=2.5cm]{geometry}
\usepackage{enumitem}
\usepackage{fancyhdr}
\usepackage[fleqn]{amsmath}
\usepackage{amssymb}
\usepackage{amsthm}
\usepackage{xfrac}
\usepackage[hyphens]{url}

\newcommand{\abs}[1]{\left|#1\right|}
\newcommand{\func}[3]{#1 : #2 \rightarrow #3}
\newcommand{\functoR}[2]{#1 : #2 \rightarrow \mathbb{R}}
\newcommand{\functoN}[2]{#1 : #2 \rightarrow \mathbb{N}}
\newcommand{\funcR}[1]{#1 : \mathbb{R} \rightarrow \mathbb{R}}

\newcommand{\limn}{\lim_{n \to \infty}}
\newcommand{\limsupn}{\limsup_{n \to \infty}}
\newcommand{\liminfn}{\liminf_{n \to \infty}}
\newcommand{\limx}[1]{\lim_{x \to #1}}

\newcommand{\tounif}{\xrightarrow{unif.}}

\newcommand{\DS}{\displaystyle}
\newcommand{\bb}[1]{\mathbb{#1}}
\newcommand{\sumn}[1]{\sum_{k=1}^n #1}
\newcommand{\series}[1]{\sum_{k=1}^\infty #1}
\newcommand{\seriec}[1]{\sum_{k=m}^n #1}
\newcommand{\pseries}[1]{\sum_{n=0}^\infty #1}

\newcommand{\intcc}[1]{\left[#1\right]}
\newcommand{\intoc}[1]{\left(#1\right]}
\newcommand{\intco}[1]{\left[#1\right)}
\newcommand{\intoo}[1]{\left(#1\right)}

\newcommand{\N}{\mathbb{N}}
\newcommand{\Z}{\mathbb{Z}}
\newcommand{\Q}{\mathbb{Q}}
\newcommand{\R}{\mathbb{R}}

\newcommand{\Ep}{\varepsilon}

\newcommand{\separator}{\noindent\makebox[\linewidth]{\rule{\textwidth}{0.4pt}}}
\newcommand{\pseparator}{\noindent\makebox[\linewidth]{\rule{\textwidth}{1pt}}}

\newcommand{\Def}{\paragraph{Definition}}
\newcommand{\Proposition}{\paragraph{Proposition}}
\newcommand{\Theorem}{\paragraph{Theorem}}
\newcommand{\Lemma}{\paragraph{Lemma}}
\newcommand{\Remark}{\paragraph{Remark}}
\newcommand{\Corollary}{\paragraph{Corollary}}
\newcommand{\Proof}{\paragraph{Proof}}
\newcommand{\Example}{\paragraph{Example}}

\newcommand{\COURSE}{Calculus}
\newcommand{\ASSIGNMENT}{Lecture Notes}
\newcommand{\SEMESTER}{Spring Semester}
\newcommand{\AUTHOR}{Amedeo Zucchetti}

%\setlength{\mathindent}{0cm}


%------------------------------------------------------------------------------%
% HEADER / FOOTER
%------------------------------------------------------------------------------%

\pagestyle{fancy}
\fancyhf{}
\lhead{\leftmark}
\rhead{\MakeUppercase{\COURSE}}
\cfoot{\thepage}

\begin{document}


%------------------------------------------------------------------------------%
% HEAD
%------------------------------------------------------------------------------%

\begin{center}

\thispagestyle{empty}

\textbf{\LARGE \COURSE \ -- \ASSIGNMENT} \vspace{.5cm}

Amedeo Zucchetti \vspace{.3cm}

\today
\vspace{.3cm}

Based on the course of Prof. Michael Bronstein at USI \vspace{3cm}

\end{center}

\tableofcontents

\newpage

%------------------------------------------------------------------------------%
% SETS, GROUPS AND FIELDS
%------------------------------------------------------------------------------%
\section{Sets, groups and fields}

%%BEGIN_BLOCK-----------------------------------------------------------------%%
	\Def Natural numbers $\N$
	\begin{enumerate}[label=(\roman*)]
		\item $1 \in \N$

		\item $n \in \N \Rightarrow n+1 \in \N$ ($n+1$ is the successor of $n$)

		\item $\nexists n \in \N : n+1 = 1$ (no number is predecessor of 1)

		\item $m, n \in \N$ and $m+1 = n+1 \Rightarrow m = n$

		\item $A \subseteq \N$, $n \in A$ and $n+1 \in A \Rightarrow A = \N$
	\end{enumerate}
%%END_BLOCK-------------------------------------------------------------------%%

%%BEGIN_BLOCK-----------------------------------------------------------------%%
	\Def Group
\\A set $X$ and an operation $\circ$ form a group $(X, \circ)$ if the following
	rules are satisfied for all $a, b, c \in X$:
	\begin{enumerate}[label=(\roman*)]
		\item Closure: $a \circ b \in X$

		\item Associativity: $(a \circ b) \circ c = a \circ (b \circ c)$

		\item Identity: $\exists! 0 \in X : a \circ 0 = 0 \circ a = a$

		\item Inverse: $\exists! (-a) \in X : a \circ (-a) = (-a) \circ a = 0$
	\end{enumerate}
	The group $(X, \circ)$ is abellian if the following rule is satisfied too:
	\begin{enumerate}[label=(\roman*)]
		\setcounter{enumi}{4}
		\item Commutativity: $a \circ b = b \circ a$
	\end{enumerate}

%%BEGIN_EXAMPLE
	\Example
	\begin{enumerate}[label=(\arabic*)]
		\item $(\Z_2, \oplus)$ is an abellian group (where $\Z_2 = \{ 0, 1 \}$ and
		$\oplus$ is exclusive or):
		\begin{enumerate}[label=(\roman*)]
			\item Closure: $0 \oplus 0 = 0$, $0 \oplus 1 = 1$, $1 \oplus 0 = 1$,
			$1 \oplus 1 = 0$
			\item Associativity: we have two elements, so we don't need to prove it
			\item Identity: $0 \Rightarrow 0 \oplus 0 = 0$, $1 \oplus 0 = 1$,
			$0 \oplus 1 = 1$
			\item Inverse: $(-1) = 1, (-0) = 0 \Rightarrow 1 \oplus 1 = 0$,
			$0 \oplus 0 = 0$
			\item Commutativity: $1 \oplus 0 = 1 = 0 \oplus 1$
		\end{enumerate}

		\item $(\N, +)$ is not a group, since it doesn't have the identity element.
	\end{enumerate}
%%END_EXAMPLE
%%END_BLOCK-------------------------------------------------------------------%%

%%BEGIN_BLOCK-----------------------------------------------------------------%%
	\Def Field
\\Given a set $X$, then $(X, +, \cdot)$ is a field if it satisfies the following
	properties for all $a, b, c \in X$:
	\begin{enumerate}[label=(\roman*)]
		\item
		$a + b \in X
	\\a \cdot b \in X$

		\item
		$(a + b) + c = a + (b + c)
	\\(a \cdot b) \cdot c = a \cdot (b \cdot c)$

		\item
		$\exists! 0 \in X : a + 0 = 0 + a = a
	\\\exists! 1 \in X : a \cdot 1 = 1 \cdot a = a$

		\item
		$\exists! (-a) \in X : a + (-a) = (-a) + a = 0
	\\\forall a \neq 0, \exists! a^{-1} : a \cdot a^{-1} = a^{-1} \cdot a = 1$

		\item
		$a + b = b + a
	\\a \cdot b = b\cdot a$

		\item
		$a \cdot (b + c) = a \cdot b + a \cdot c$
	\end{enumerate}

%%BEGIN_EXAMPLE
	\Example Application of field axioms
	\begin{enumerate}[label=(\arabic*)]
		\item If $a \cdot b = 0 \Rightarrow$ either $a$ or $b$ are equal to 0. We
		suppose $b \neq 0$:
	\\$0 = 0 \cdot b^{-1} = (a \cdot b) \cdot b^{-1} = a \cdot (b \cdot b^{-1}) =
		a \cdot 1 = a \Rightarrow a = 0$ (the same can be done supposing $a \neq 0$)

		\item $a \cdot 0 = 0$:
	\\$a \cdot 0 = a \cdot (0 + 0) = a \cdot 0 + a \cdot 0 \Rightarrow$ we
		subtract from both sides $(- \ a \cdot 0)$:
	\\$a \cdot 0 + (- \ a \cdot 0) = a \cdot 0 + a \cdot 0 + (- \ a \cdot 0) \iff
		0 = a \cdot 0$
	\end{enumerate}
%%END_EXAMPLE
%%END_BLOCK-------------------------------------------------------------------%%

%%BEGIN_BLOCK-----------------------------------------------------------------%%
	\Def $\Q = \{ \frac{p}{q} : p, q \in \Z, q \neq 0 \}$

	\Remark $(\Q, +, \cdot)$ is a field.
%%END_BLOCK-------------------------------------------------------------------%%

%%BEGIN_BLOCK-----------------------------------------------------------------%%
	\Def Ordered field
\\Let $\leq$ be an order relation. Then the field $(X, +, \cdot, \leq)$ is an
	ordered field if the following properties are satisfied for $a, b, c\in X$:
	\begin{enumerate}[label=(\roman*)]
		\item Either $a \leq b$ or $b \leq a$

		\item If $a \leq b$ and $b \leq a$, then $a = b$

		\item If $a \leq b$ and $b \leq c$, then $a \leq c$

		\item If $a \leq b$, then $a + c \leq b + c$

		\item If $a \leq b$ and $0 \leq c$, then $a \cdot c \leq b \cdot c$
	\end{enumerate}

	\Example Application of the order axioms
\\Let's take $(\Q, +, \cdot, \leq)$, $a, b \in \Q$. We want to show that if $a
	\leq b$, then $(-b) \leq (-a)$:
\\$a \leq b \iff a + ((-a) + (-b)) \leq b + ((-a) + (-b)) \iff (a + (-a)) + (-b)
 	\leq (-a) + (b + (-b))
\\\iff (-b) + 0 \leq (-a) + 0 \iff (-b) \leq (-a)$
%%END_BLOCK-------------------------------------------------------------------%%

%%BEGIN_BLOCK-----------------------------------------------------------------%%
	\Def A set $A$ is countably infinite if it exists a function
	$\functoN{f}{A}$ bijective.

	\Remark Let $A, B$ sets:
	\begin{itemize}
		\item If $|A| = |B| \iff$ exists a bijection between $A$ and $B$

		\item If $|A| \leq |B| \iff$ exists an injection from $A$ to $B$

		\item If $|A| < |B| \iff$ exists an injection, but not a bijection
	\end{itemize}
%%END_BLOCK-------------------------------------------------------------------%%

%%BEGIN_BLOCK-----------------------------------------------------------------%%
	\Proposition $\Z$ is countably infinite

%%BEGIN_PROOF
	\Proof We can arrange $\Z$ and $\Z$ in the following way:
	\begin{equation*}
		\begin{array}{lcrccccccccl}
			\bb{N} & = & \{ & 1, & 2, & 3, & 4, & 5, & 6, & 7, & \hdots & \} \\
			\bb{Z} & = & \{ & 0, & 1, & -1,& 2, & -2,& 3, & -3,& \hdots & \} \\
		\end{array}
	\end{equation*}
	We can take the function $\functoN{f}{\Z}$ such that:
	\begin{equation*}
		f(x) = \begin{cases}
			0 									& \text{if } x = 1					\\
			\frac{x}{2} 				& \text{if } x \text{ even}	\\
			-\frac{(x - 1)}{2} 	& \text{if } x \text{ odd}
		\end{cases}, \quad
		f^{-1}(x) = \begin{cases}
			1 			& \text{if } x = 0	\\
			2x 			& \text{if } 0 < x 	\\
			-2x+1 	& \text{if } x < 0
		\end{cases}
	\end{equation*}
	$f$ is bijective, thus $\Z$ is countably infinite.
\\\qed
%%END_PROOF
%%END_BLOCK-------------------------------------------------------------------%%

%%BEGIN_BLOCK-----------------------------------------------------------------%%
	\Proposition $\Q$ is countably infinite

%%BEGIN_PROOF
	\Proof Idea of the proof. We can arrange $\N$ and $\Q$ as such:
	\begin{equation*}
		\begin{array}{lcrcccccccccl}
			\bb{N} & = & \{ & 1, & 2, & 3, & 4, & 5, & 6, & 7, & 8, & \hdots & \} \\
			\bb{Q} & = & \{ & \frac{0}{1}, & \frac{1}{1}, & -\frac{1}{1}, &
			\frac{1}{2}, & -\frac{1}{2}, & \frac{2}{1},& -\frac{2}{1}, & \frac{1}{3},
			& \hdots & \} \\
		\end{array}
	\end{equation*}
	Similarly to the proof for $\Z$, we can find a bijection between $\N$
	and $\Q$.
\\\qed
%%END_PROOF
%%END_BLOCK-------------------------------------------------------------------%%

%%BEGIN_BLOCK-----------------------------------------------------------------%%
	\Proposition $\R$ is not countable

%%BEGIN_PROOF
	\Proof Idea of the proof. Let $x \in {[}0,1{[}$. Each $x$ can be written as an
	infinite succession of digits:
\\\begin{tabular}{l|l}
		1 & 0.\textbf{1}786... \\
		2 & 0.3\textbf{9}09... \\
		3 & 0.45\textbf{0}0... \\
		4 & 0.097\textbf{2}... \\
		... & ...
	\end{tabular}
\\We can construct a new number, taking a digit from each number (each at a
	different position) and increment it by 1. This way, the new number will be
	different from any other in the list in the position from where the digit was
	taken. In our example, the new number would be 0.\textbf{2013}...
\\Since there is one more number than those in the list, then $|\N| < |\R|$, so
	there is no bijection, and $\R$ is uncountable.
\\\qed
%%END_PROOF
%%END_BLOCK-------------------------------------------------------------------%%

%%BEGIN_BLOCK-----------------------------------------------------------------%%
	\Proposition $|\R| = |\R^2|$
%%END_BLOCK-------------------------------------------------------------------%%

%%BEGIN_BLOCK-----------------------------------------------------------------%%
	\Def Power set
\\Let $A$ be a set. The power set of $A$ is $2^A = \{ A' : A' \subseteq A \}$,
	the set containing all subsets of $A$. $|2^A| = 2^{|A|}$
%%END_BLOCK-------------------------------------------------------------------%%

%%BEGIN_BLOCK-----------------------------------------------------------------%%
	\Proposition $|2^{\N}| = |\R|$
%%END_BLOCK-------------------------------------------------------------------%%

%%BEGIN_BLOCK-----------------------------------------------------------------%%
	\Proposition $\sqrt{2} \notin \Q$

%%BEGIN_PROOF
	\Proof By contradiction
\\We suppose $\sqrt{2} \in \Q$, this means there exists $a, b \in \Z$, $b \neq
	0$ and greatest common divisor of $a$ and $b$ is 1, such
	that $\sqrt{2} = \frac{a}{b}$:
	\begin{equation*}
		\sqrt{2} = \frac{a}{b} \iff 2 = \frac{a^2}{b^2} \iff 2b^2 = a^2
	\end{equation*}
	This means $a^2$ is even (and $a$ is even), then it exists $c$ such that $a =
	2c$:
	\begin{equation*}
		2b^2 = a^2 \iff 2b^2 = (2c)^2 = 4c^2 \iff b^2 = 2c^2
	\end{equation*}
	This means $b^2$, and $b$, are even. But if both $a$ and $b$ are even, then
	the greatest common divisor of $a$ and $b$ is not 1, contradiction.
\\We can conclude that $\sqrt{2} \notin \Q$.
\\\qed
%%END_PROOF
%%END_BLOCK-------------------------------------------------------------------%%

%%BEGIN_BLOCK-----------------------------------------------------------------%%
	\Def Bounds
\\Let $A, X$ be sets, such that $A \subseteq X$, and $x \in X$, then:
	\begin{itemize}
		\item $x$ is upper bound of $A$ if $a \leq x$, for all $a \in A$
		\item $x$ is lower bound of $A$ if $x \leq a$, for all $a \in A$
	\end{itemize}
%%END_BLOCK-------------------------------------------------------------------%%

%%BEGIN_BLOCK-----------------------------------------------------------------%%
	\Def Supremum and infimum
\\Let $A$ be a set:
	\begin{itemize}
		\item The supremum is the smallest upper bound of $A$
		\item The infimum is the greatest lower bound of $A$
	\end{itemize}
%%END_BLOCK-------------------------------------------------------------------%%

%%BEGIN_BLOCK-----------------------------------------------------------------%%
	\Def Maximum and minimum
\\Let $A$ be a set:
	\begin{itemize}
		\item The maximum is the biggest element of $A$ (if $\sup(A) \in A$, then
		$\max(A) = \sup(A)$)
		\item The minimum is the smallest element of $A$ (if $\inf(A) \in A$, then
		$\min(A) = \inf(A)$)
	\end{itemize}
%%END_BLOCK-------------------------------------------------------------------%%

\newpage
%------------------------------------------------------------------------------%
% SPACES
%------------------------------------------------------------------------------%
\section{Spaces}

%%BEGIN_BLOCK-----------------------------------------------------------------%%
	\Def Topology
\\Let $X$ be a set. Then $\tau \subseteq 2^X$ is a topology if:
	\begin{enumerate}[label=(\roman*)]
		\item $X \in \tau$
		\item $\emptyset \in \tau$
		\item $A_\alpha \in \tau$, then $\displaystyle \bigcup_\alpha A_\alpha \in
		\tau$ (the union of any element of $\tau$ is also contained in $\tau$)
		\item $A_i \in \tau$, then $\displaystyle \bigcap_{i = 1}^n A_i \in \tau$
		(any finite intersection of elements of $\tau$ is also contained in $\tau$)
	\end{enumerate}

%%BEGIN_EXAMPLE
	\Example Let $X = \{ 1, 2, 3, 4 \}$
	\begin{enumerate}[label=(\arabic*)]
		\item $\tau = \{ \emptyset, X \}$ is topology:
	\\$\emptyset \cup X = X \in \tau$
	\\$\emptyset \cap X = \emptyset \in \tau$

		\item $\tau = \{ \emptyset, \{ 2 \}, \{ 2, 3 \}, X \}$ is topology:
	\\The cases with $\emptyset$ and $X$ are trivial
	\\$\{ 2 \} \cup \{ 2, 3 \} = \{ 2, 3 \} \in \tau$
	\\$\{ 2 \} \cap \{ 2, 3 \} = \{ 2 \} \in \tau$

		\item $A = \{ \emptyset, \{ 2 \}, \{ 3 \}, X \}$ is not a topology:
	\\$\{ 2 \} \cup \{ 3 \} = \{ 2, 3 \} \notin A$
	\end{enumerate}
%%END_EXAMPLE
%%END_BLOCK-------------------------------------------------------------------%%

%%BEGIN_BLOCK-----------------------------------------------------------------%%
	\Def Topological space
\\Let $X$ be a set, $\tau$ a topology, then $(X, \tau)$ is a topological space.
%%END_BLOCK-------------------------------------------------------------------%%

%%BEGIN_BLOCK-----------------------------------------------------------------%%
	\Def Neighborhood in a topological space $(X, \tau)$
\\A set $N$ is a neighborhood of $x \in X$ if there exists a set $U \in \tau$
	such that $x \in U$ and $U \subseteq N$.
%%END_BLOCK-------------------------------------------------------------------%%

%%BEGIN_BLOCK-----------------------------------------------------------------%%
	\Def Metric
\\Let $X$ be a set, $x, y, z \in X$. The function $d : X \times X \rightarrow
	\R$ is a metric if:
	\begin{enumerate}[label=(\roman*)]
		\item $d(x,y) = d(y,x)$
		\item $d(x,y) = 0 \iff x = y$
		\item $d(x,z) \leq d(x,y) + d(x,z)$
	\end{enumerate}

%%BEGIN_EXAMPLE
	\Example
	\begin{enumerate}[label=(\arabic*)]
		\item $d(x,y) = \abs{x - y}$
		\item $\displaystyle d(x,y) = ||x - y||_2 =
		\left(\sum_{i = 1}^n (x_i - y_i)^2\right)^{\frac{1}{2}}$
	\end{enumerate}
%%END_EXAMPLE
%%END_BLOCK-------------------------------------------------------------------%%

%%BEGIN_BLOCK-----------------------------------------------------------------%%
	\Def Metric space
\\Let $X$ be a set, $d$ be a metric, then $(X, d)$ is a metric space.
%%END_BLOCK-------------------------------------------------------------------%%

%%BEGIN_BLOCK-----------------------------------------------------------------%%
	\Def Ball in a metric space $(X, d)$
\\$B_r(x) = \{ y \in X : d(x,y) < r \}$ is a ball of center $x$ and radius $r$.
	$B_r(x)$ is subset of $X$.
%%END_BLOCK-------------------------------------------------------------------%%

%%BEGIN_BLOCK-----------------------------------------------------------------%%
	\Def Open set in a topological space $(X, \tau)$
\\A set $U$ is open in $(X, \tau)$ if $U \in \tau$.
%%END_BLOCK-------------------------------------------------------------------%%

%%BEGIN_BLOCK-----------------------------------------------------------------%%
	\Def Open set in a metric space $(X, d)$
\\A set $U$ is open in $(X, d)$ if for all $x \in U$ exists $\Ep > 0$ such
	that $B_\Ep(x) \subseteq U$.
%%END_BLOCK-------------------------------------------------------------------%%

%%BEGIN_BLOCK-----------------------------------------------------------------%%
	\Def $C \subseteq X$ is closed if $X\setminus C$ is open. A set is closed if
	its complement is open.
%%END_BLOCK-------------------------------------------------------------------%%

%%BEGIN_BLOCK-----------------------------------------------------------------%%
	\Proposition Let $S = (X, x)$ be a space ($x$ a metric or a topology), then:
	\begin{enumerate}[label=(\roman*)]
		\item $X$ is open in $S$
		\item $\emptyset$ is open in $S$
		\item For all $A_\alpha$ open in $S$, then $\displaystyle \bigcup_\alpha
		A_\alpha$ is open in $S$ (any union of any open set is also open)
		\item For all $A_i$ open in $S$, then $\displaystyle \bigcap_{i=1}^n
		A_i$ is open in $S$ (any finite intersection of any open set is also open)
	\end{enumerate}

%%BEGIN_PROOF
	%TODO: add proof (topological or metric?)
%%END_PROOF
%%END_BLOCK-------------------------------------------------------------------%%


%------------------------------------------------------------------------------%
% SEQUENCES
%------------------------------------------------------------------------------%
\section{Sequences}

%%BEGIN_BLOCK-----------------------------------------------------------------%%
	\Def Sequence
\\A sequence $(x_n)$ is a function $\func{x}{\N}{X}$, where $x(n) = x_n$.
\\The elements of a sequence can be listed in an ordered set with repetition:
	$(x_n) = (x_1, x_2, x_3, x_4, \hdots)$

%%BEGIN_EXAMPLE
	\Example
	\begin{itemize}
		\item $a_n = n \Rightarrow (a_n) = (1, 2, 3, 4, 5, \hdots)$
		\item $b_n = \frac{1}{n} \Rightarrow (b_n) = (1, \frac{1}{2}, \frac{1}{3},
		\hdots)$
		\item $c_n = (-1)^n \Rightarrow (c_n) = (1, -1, 1, -1, \hdots)$
	\end{itemize}
%%END_EXAMPLE
%%END_BLOCK-------------------------------------------------------------------%%

%%BEGIN_BLOCK-----------------------------------------------------------------%%
	\Def Cauchy sequence
\\A sequence $(x_n)$ is a Cauchy sequence if for all $\Ep > 0$ exists
	$N_\Ep$ such that $d(x_n,x_m) < \Ep$, for all $n, m \geq N_\Ep$.
\\That is, starting from an index $N_\Ep$ all values $x_n$ are contained in
	an interval ${[}x_{N_\Ep} - \Ep, x_{N_\Ep} + \Ep{]}$.

%%BEGIN_EXAMPLE
	\Example $x_n = \frac{1}{n}$ in $(\R, d)$, where $d(x,y) = \abs{x - y}$. We
	have to find, for each $\Ep$, an $N$ that satisfies the definition of
	Cauchy sequence.
\\Let's take $N \leq n \leq m$. Thanks to the triangle inequality, we can first
	find that:
	\begin{equation*}
		d(x_n,x_m) = \abs{x_n - x_m} \leq \abs{x_n} + \abs{-x_m} =
		\abs{x_n} + \abs{x_m} = \abs{\frac{1}{n}} + \abs{\frac{1}{m}} =
		\frac{1}{n} + \frac{1}{m}
	\end{equation*}
	Since $N \leq n \leq m$, then we have $\frac{1}{m} \leq \frac{1}{n} \leq
	\frac{1}{N}$:
	\begin{equation*}
		\frac{1}{n} + \frac{1}{m} \leq
		\frac{1}{N} + \frac{1}{N} = \frac{2}{N}
	\end{equation*}
	Now, in order to satisfy the definition we must have $\frac{2}{N} \leq
	\Ep$, thus $\frac{2}{\Ep} \leq N$. This means, starting from $N =
\frac{2}{\Ep}$ all $d(x_n,x_m)$ wil be smaller than $\Ep$. In fact,
	if we take the previous inequality:
	\begin{equation*}
		\abs{\frac{1}{n} - \frac{1}{m}} \leq
		\frac{2}{N} = \frac{2}{\frac{2}{\Ep}} = \Ep
	\end{equation*}
	Note that it is not important if $d(x,y) < \Ep$ or $d(x,y) \leq \Ep$.
%%END_EXAMPLE
%%END_BLOCK-------------------------------------------------------------------%%

%%BEGIN_BLOCK-----------------------------------------------------------------%%
	\Def Convergence in a metric space $(X, d)$
\\A sequence $(x_n)$ converges to a limit $x$ if for all $\Ep > 0$ exists
	$N_\Ep$ such that $d(x_n,x) < \Ep$, for all $n \geq N_\Ep$.

%%BEGIN_EXAMPLE
	\Example $x_n = \frac{1}{n}$ in $\R$ converges to 0. We take $N \leq n$ and
	$N = \frac{1}{\Ep}$:
	\begin{equation*}
		\abs{x_n - 0} = \abs{x_n} = \abs{\frac{1}{n}} = \frac{1}{n} \leq \frac{1}{N}
		= \frac{1}{\frac{1}{\Ep}} = \Ep
	\end{equation*}
%%END_EXAMPLE
%%END_BLOCK-------------------------------------------------------------------%%

%%BEGIN_BLOCK-----------------------------------------------------------------%%
	\Def Convergence in a topological space $(X, \tau)$
\\A sequence $(x_n)$ converges to a limit $x$ if for all $U \in \tau$ such that
	$x \in U$, it exists $N_U$ such that $x_n \in U$, for all $n \geq N_U$.
\\That is, $x$ is a limit of a sequence, if all sets of $\tau$ that contain $x$
	also contain the tail of the sequence.

%%BEGIN_EXAMPLE
	\Example Let's take $X = \{ a, b, c \}$, $(x_n) = (a, b)$ and $\tau =
	\{ \emptyset, \{ a \}, X \}$.
	\begin{itemize}
		\item $a$ is not a limit of $(x_n)$, in fact $\{ a \}$ contains $a$, but
		doesn't contain the tail of $(x_n)$
		\item $b$ and $c$ are limits of $(x_n)$, in fact $b \in X$ and $c \in X$,
		and $X$ contains $(x_n)$ (and its tail too)
	\end{itemize}
%%END_EXAMPLE
%%END_BLOCK-------------------------------------------------------------------%%

%%BEGIN_BLOCK-----------------------------------------------------------------%%
	\Proposition $x_n \to x$ in $(X, d) \iff$ for all $U \subseteq X$ open exists
	$N_U$ such that $x_n \in U$, for all $n \geq N_U$.
	%If a sequence converges in metric, then converges it in a topological space?

%%BEGIN_PROOF
	\Proof Let $U \subseteq X$ open, $x \in U$, it exists $\Ep > 0$ such that
	$B_\Ep(x) \subseteq U$.
	\begin{itemize}
		\item[$\Rightarrow$] Since $x_n$ converges, then $d(x_n,x) < \Ep$, for
		all $\Ep$. This means $x \in B_\Ep(x) \subseteq U$, thus $x \in U$
		\item[$\Leftarrow$] $x \in B_\Ep(x)$ open. This mean it exists $N$ such
		that all $x_n \in B_\Ep(x)$, for all $n \geq N$. We can conclude that
		$d(x_n,x) < \Ep$.
	\end{itemize}
	\qed
%%END_PROOF
%%END_BLOCK-------------------------------------------------------------------%%

%%BEGIN_BLOCK-----------------------------------------------------------------%%
	\Theorem If a sequence converges to a limit in a metric space, then the limit
	is unique.

%%BEGIN_PROOF
	\Proof Let's suppose $x_n \to x$ and $x_n \to x'$.
	It exists $N$ such that for $n geq N$, $d(x_n,x) < \Ep$ and It exists
	$N'$ such that for $n \geq N'$, $d(x_n,x') < \Ep$. We take $n \geq
	\max\{ N, N' \}$. Now we have $0 \leq d(x,x') \leq d(x,x_n) + d(x_n,x') <
	2\Ep$.
\\Since $\Ep$ is arbitrarily small, then $d(x,x') \leq 0$. Now, we have
	$0 \leq d(x,x') \leq 0 \Rightarrow d(x,x') = 0 \Rightarrow x = x'$.
	\qed
%%END_PROOF

	\Remark This isn't true in a topological space. In a topological space, a
	sequence can converge to multiple limits.
%%END_BLOCK-------------------------------------------------------------------%%

%%BEGIN_BLOCK-----------------------------------------------------------------%%
	\Proposition $x_n \to x$ in $(X,d)$ metric space, then for all $y \in X$,
	$d(x_n,y) \to d(x,y)$.

%%BEGIN_PROOF
	%???
%%BEGIN_PROOF
%%END_BLOCK-------------------------------------------------------------------%%

%%BEGIN_BLOCK-----------------------------------------------------------------%%
	\Proposition Properties of real sequences
	\\For all $(x_n), (y_n)$ such that $x_n \to x$, $y_n \to y$, we have the
		following properties:
		\begin{enumerate}[label=(\roman*)]
			\item $ \displaystyle
			\limn (\alpha x_n + \beta y_n) = \alpha \limn x_n + \beta \limn y_n $

			\item $ \displaystyle
			\limn x_n x_y = \limn x_n \limn y_n $

			\item $ \displaystyle
			\limn \frac{x_n}{x_y} = \frac{\limn x_n}{\limn y_n} $
		\end{enumerate}

%%BEGIN_PROOF
	\Proof
		\begin{enumerate}[label=(\roman*)]
			\item $ \displaystyle
			\forall \Ep > 0, \quad
			\exists N : \abs{x_n - x} < \frac{\Ep}{2\abs{\alpha}} = \Ep', \quad
			\exists N' : \abs{y_n - y} < \frac{\Ep}{2\abs{\beta}} = \Ep''
			\rightarrow $ we take $ n = \max \{ N, N'\} $.
			\begin{equation*}
				\abs{(\alpha x_n + \beta y_n) - (\alpha x + \beta y)} =
				\abs{\alpha (x_n - x) + \beta (y_n - y)} \leq
				\abs{\alpha (x_n - x)} + \abs{\beta (y_n - y)} =
			\end{equation*}
			\begin{equation*} =
				\abs{\alpha} \abs{(x_n - x)} + \abs{\beta} \abs{(y_n - y)} <
				\abs{\alpha} \Ep' + \abs{\beta} \Ep'' =
				\frac{\Ep}{2} + \frac{\Ep}{2} =
				\Ep
			\end{equation*}

			\item[(i-ii)] Similar to previous demonstration.
		\end{enumerate}
		\qed
%%END_PROOF

%%BEGIN_EXAMPLE
	\Example
		\begin{enumerate}[label=(\arabic*)]
			\item
			Knowing that $\frac{1}{n} \to 0$, show that $\frac{1}{n^2} \to 0$.
			\begin{equation*}
				\limn \frac{1}{n^2} = \limn \frac{1}{n} \cdot \limn \frac{1}{n} = 0 \cdot 0 = 0
			\end{equation*}

			\item
			Find the limit of the sequence $\frac{2n^2 - 3n + 2}{n^2 + n -1}$.
			\begin{equation*}
				\limn \frac{2n^2 - 3n + 2}{n^2 + n -1} =
				\frac{\lim (2n^2 - 3n + 2)}{\lim (n^2 + n -1)} =
				\frac{\lim 2 - \lim \frac{3}{n} + \lim \frac{2}{n^2}}{\lim 1 +
				\lim \frac{1}{n} - \lim \frac{1}{n^2}} =
				\frac{2 - 0 + 0}{1 + 0 - 0} = 2
			\end{equation*}
		\end{enumerate}
%%END_EXAMPLE
%%END_BLOCK-------------------------------------------------------------------%%

%%BEGIN_BLOCK-----------------------------------------------------------------%%
	\Def A sequence $(x_n)$ is bounded if exists $c$ such that $\abs{s_n} \leq c$.

%%BEGIN_EXAMPLE
	\Example
	\begin{itemize}
		\item $a_n = \frac{1}{n}$ is bounded. We can take $c = 1$,
		$\abs{\frac{1}{n}} = \frac{1}{n} \leq 1$, being $a_1 = 1$ the $\sup(a_n)$.
		\item $b_n = (-1)^n$ is bounded. In fact, the values of the sequence are
		always 1 and -1. If we take $c = 1$, then $\abs{b_n} = 1 \leq 1$.
		\item $x_n = n$ is not bounded, we can prove it by contradiction.
	\\We suppose it exists a $c$ such that $\abs{x_n} \leq c$. If we take $x_{c+1}
		= c + 1$, we have $c + 1 \leq c \iff 0 \leq 1$ contradiction. This means
		$x_n$ is not bounded.
	\end{itemize}
%%END_EXAMPLE
%%END_BLOCK-------------------------------------------------------------------%%

%%BEGIN_BLOCK-----------------------------------------------------------------%%
	\Theorem Convergent real sequences are bounded (not the opposite).

%%BEGIN_PROOF
	\Proof Since $x_n \to x$, it exists $N$ such that for all $n \geq N$,
	$\abs{x_n - x} < \Ep$. By triangle inequality we have:
	\begin{equation*}
		\abs{x_n} = \abs{x_n - x + x} \leq \abs{x_n - x} + \abs{x} < \Ep + \abs{x}
	\end{equation*}
	We choose $c = \max\{\abs{x_1}, \hdots, \abs{x_{N-1}}, \abs{x} + \Ep\}$,
	then $\abs{x_n} < c$ for each $n$.
\\\qed
%%END_PROOF
%%END_BLOCK-------------------------------------------------------------------%%

%%BEGIN_BLOCK-----------------------------------------------------------------%%
	\Def Monotonic sequences
	\begin{itemize}
		\item $(x_n)$ is monotonic increasing if $x_n \leq x_{n+1}$ for all $n$
		\item $(x_n)$ is monotonic decreasing if $x_{n+1} \leq x_n$ for all $n$
	\end{itemize}

%%BEGIN_EXAMPLE
	\Example
	\begin{itemize}
		\item $a_n = \frac{1}{n}$ is monotonic decreasing. In fact, $\frac{1}{n+1}
		\leq \frac{1}{n}$, then $a_{n+1} \leq a_n$.
		\item $b_n = n$ is monotonic increasing. In fact, $b_n = n$ and $b_{n+1} =
		n + 1$. Since $n \leq n + 1$, then $b_n \leq b_{n+1}$.
		\item $c_n = (-1)^n$ is not monotonic. We can take $n = 1$, then $a_1 \leq
		a_2 \nleq a_3$.
	\end{itemize}
%%END_EXAMPLE
%%END_BLOCK-------------------------------------------------------------------%%

%%BEGIN_BLOCK-----------------------------------------------------------------%%
	\Theorem If a sequence monotonic and bounded $\Rightarrow$ convergent

%%BEGIN_PROOF
	\Proof $(x_n)$ increasing and bounded, let $c = \sup(x_n)$. For all $\Ep
	> 0$ exists $N$ such that $c - \Ep < x_N$. Since $(x_n)$ increasing, for
	all $n \geq N$, $x_N \leq x_n \leq c$.
	\begin{equation*}
		c - \Ep < x_n \leq c \iff -\Ep < x_n - c \leq 0 < \Ep \iff
		\abs{x_n - c} < \Ep
	\end{equation*}
	The last inequality implies convergence. Similarly, the theorem can be proven
	for decreasing sequences.
\\\qed
%%END_PROOF
%%END_BLOCK-------------------------------------------------------------------%%

%%BEGIN_BLOCK-----------------------------------------------------------------%%
	\Def Limit superior and inferior of $(x_n)$
	\begin{itemize}
		\item $\limsup_{n \to \infty} x_n = \limn \sup\{x_k : k \geq n\}$
		\item $\liminf_{n \to \infty} x_n = \limn \inf\{x_k : k \geq n\}$
	\end{itemize}
%%END_BLOCK-------------------------------------------------------------------%%

%%BEGIN_BLOCK-----------------------------------------------------------------%%
	\Def Subsequence
\\$(x_{n_k}) \subseteq (x_n)$ is a subsequence of $(x_n)$. Only some terms of a
	sequence are part of a subsequence.

%%BEGIN_EXAMPLE
	\Example $x_n = (-1)^n \cdot n$. We take $k = 2n$, then the subsequence
	$(x_{n_k}) = (x_{2n})$ of $(x_n)$ takes all the even indexes $n$ of $(x_n)$:
	\begin{equation*}
		\begin{array}{lcrcccccccl}
			(x_n)		& = & ( & -1, & 2, & -3, & 4, & -5, & 6, & \hdots & )\\
			(x_2n)	& = & ( & 		& 2, &		 & 4, &			& 6, & \hdots & )
		\end{array}
	\end{equation*}
%%END_EXAMPLE
%%END_BLOCK-------------------------------------------------------------------%%

%%BEGIN_BLOCK-----------------------------------------------------------------%%
	\Theorem If $x_n \to x$ $\Rightarrow x_{n_k} \to x$. If a sequence converges,
	all subsequences converge to the same limit.

%%BEGIN_PROOF
	\Proof $k \leq n_k$ (it can be proved by induction) and $d(x_k,x) < \Ep$.
	Since $N \leq k \leq n_k$, then $d(x_{n_k},x) \leq d(x_k,x) < \Ep$. This
	means the subsequence converges to $x$.
\\\qed
%%END_PROOF
%%END_BLOCK-------------------------------------------------------------------%%

%%BEGIN_BLOCK-----------------------------------------------------------------%%
	\Def $x_n$ is a dominant term if $x_m < x_n$ for all $n < m$.
%%END_BLOCK-------------------------------------------------------------------%%

%%BEGIN_BLOCK-----------------------------------------------------------------%%
	\Theorem Every sequence has a monotonic subsequence.

%%BEGIN_PROOF
	\Proof Based on dominants terms:
	\begin{itemize}
		\item If we have infinite dominant terms, we take the decreasing subsequence
		formed by the dominant terms.
		\item If we have a finite number of dominant terms, then, after the last
		dominant term, we start taking an increasing subsequence (since, for each
		term, there will be at some point a greater term).
	\end{itemize}
	\qed
%%END_PROOF
%%END_BLOCK-------------------------------------------------------------------%%

%%BEGIN_BLOCK-----------------------------------------------------------------%%
	\Theorem Bolzano-Weierstrass
\\Every bounded sequence has a convergent subsequence.

%%BEGIN_PROOF
	\Proof We take $(x_n)$ bounded. We show it in three steps:
	\begin{itemize}
		\item $(x_n)$ has a monotonic subsequence $(x_{n_k})$
		\item Since $(x_n)$ is bounded, then $(x_{n_k})$ is bounded
		\item Since $(x_{n_k})$ is bounded and monotonic, it is convergent
	\end{itemize}
	\qed
%%END_PROOF
%%END_BLOCK-------------------------------------------------------------------%%

%%BEGIN_BLOCK-----------------------------------------------------------------%%
	\Def $X \subseteq \R^n$ is compact $\iff X$ is closed and bounded (this is not
	true for $\R^\infty$).
%%END_BLOCK-------------------------------------------------------------------%%

\newpage
%------------------------------------------------------------------------------%
% SERIES
%------------------------------------------------------------------------------%
\section{Series}

%%BEGIN_BLOCK-----------------------------------------------------------------%%
	\Def Series
\\$(x_n)$ sequence. $\DS s_n = \sumn{x_k}$ is a series (also known as
	the partial sum). A series is the summation of the terms of a sequence.
%%END_BLOCK-------------------------------------------------------------------%%

%%BEGIN_BLOCK-----------------------------------------------------------------%%
	\Def Convergence of series
\\$\DS s_n = \sumn{x_k}$ a series. $\limn s_n = \limn \sumn{x_k} = \series{x_k}$.

%%BEGIN_EXAMPLE
	\Example
	\begin{itemize}
		\item Harmonic: $\DS \series{\frac{1}{n}} = +\infty$
		\item Geometric: $\DS \series{a^n} = \begin{cases}
		\infty & \abs{a} \geq 1 \\ \frac{1}{1-a} & \abs{a} < 1\end{cases}$
		\item Exponential: $\DS \series{\frac{1}{n!}} = e$
		\item Leibniz: $\DS \series{\frac{(-1)^n}{2n-1}} = \frac{\pi}{4}$
	\end{itemize}
%%END_EXAMPLE
%%END_BLOCK-------------------------------------------------------------------%%

%%BEGIN_BLOCK-----------------------------------------------------------------%%
	\Def Absolute convergence of a series $\DS s_n = \sumn{x_k}$
\\$s_n$ converges absolutely if $\DS \series{\abs{x_k}} < \infty$.
%%END_BLOCK-------------------------------------------------------------------%%

%%BEGIN_BLOCK-----------------------------------------------------------------%%
	\Proposition Absolute convergence $\Rightarrow$ convergence. If $\DS
	\series{\abs{x_k}} < \infty$, then $\DS \series{x_k} < \infty$.

%%BEGIN_PROOF
	\Proof $\DS \series{\abs{x_k}} < \infty$ and $x_n \leq \abs{x_n}$, then $\DS
	\series{x_k} \leq \series{\abs{x_k}} < \infty$.
\\\qed
%%END_PROOF
%%END_BLOCK-------------------------------------------------------------------%%

%%BEGIN_BLOCK-----------------------------------------------------------------%%
	\Def Cauchy criterion for series
\\$\DS s_n = \sumn{x_k}$, and$\DS \series{x_k} < \infty$ is a Cauchy series if
	for all $\Ep > 0$ it exists $N$ such that:
	\begin{equation*}
		\forall N \leq m \leq n \Rightarrow \abs{s_n - s_m} =
		\abs{\sumn{x_k} - \sum_{k=1}^m x_k} = \abs{\sum_{k=m}^n x_k} < \Ep
	\end{equation*}
%%END_BLOCK-------------------------------------------------------------------%%

%%BEGIN_BLOCK-----------------------------------------------------------------%%
	\Proposition Comparison test, for $x_n, y_n$ sequences and $x_n \geq 0$
	\begin{enumerate}[label=(\roman*)]
		\item If $\DS \series{x_k} < \infty$ and $\abs{y_n} \leq x_n \DS \Rightarrow
		\series{y_k} < \infty$
		\item If $\DS \series{x_k} = +\infty$ and $x_n \leq y_n \DS \Rightarrow
		\series{y_k} = +\infty$
	\end{enumerate}

%%BEGIN_PROOF
	\Proof
	\begin{enumerate}[label=(\roman*)]
		\item $\DS \abs{\seriec{y_k}} \leq \seriec{\abs{y_k}} \leq \seriec{x_k} <
		\Ep \Rightarrow \series{y_k} < \infty$
		\item $\DS +\infty = \series{x_k} \leq \series{y_k} \Rightarrow \series{y_n}
		= +\infty$
	\end{enumerate}
	\qed
%%END_PROOF

%%BEGIN_EXAMPLE
	\Example $\DS \series \frac{1}{n^2+1} \Rightarrow \series \frac{1}{n^2+1} <
	\series \frac{1}{n^2} < \infty$, the series converges.
%%END_EXAMPLE
%%END_BLOCK-------------------------------------------------------------------%%

%%BEGIN_BLOCK-----------------------------------------------------------------%%
	\Proposition Ratio test, for $x_n$ sequence, $x_n \neq 0$ and $\DS
	s_n = \sumn{x_k}$ series:
	\begin{enumerate}[label=(\roman*)]
		\item $s_n$ converges absolutely if $\limsupn \abs{\frac{x_{n+1}}{x_n}} < 1$
		\item $s_n$ diverges if $\liminfn \abs{\frac{x_{n+1}}{x_n}} > 1$
	\end{enumerate}

%%BEGIN_EXAMPLE
	\Example
	\begin{itemize}
		\item $\DS \series\left({-\frac{1}{3}}\right)^n$:
		\begin{equation*}
			\abs{\frac{(-\frac{1}{3})^{n+1}}{(-\frac{1}{3})^n}} = \abs{-\frac{1}{3}}
			= \frac{1}{3} \Rightarrow \limsupn \frac{1}{3} = \frac{1}{3} < 1
			\Rightarrow \text{ converges absolutely}
		\end{equation*}
		\item $\DS \series \frac{n}{n^2+3}$
			\begin{itemize}
				\item Ratio test:
				\begin{equation*}
					\limsupn \abs{\frac{n+1}{(n+1)^2+3}\frac{n^2+3}{n}} =
					\limsupn \frac{n+1}{(n+1)^2+3}\frac{n^2+3}{n} = 1 \text{, no information}
				\end{equation*}
				\item Comparison test:
				\begin{equation*}
					\frac{n}{n^2+3n^2} \leq \frac{n}{n^2+3} \Rightarrow
					\frac{n}{n^2+3n^2} = \frac{n}{4n^2} = \frac{1}{4} \frac{n}{n^2}
					\Rightarrow \series \frac{1}{4} \frac{n}{n^2} \Rightarrow
					+\infty = \frac{1}{4} \series \frac{n}{n^2} \leq \series \frac{n}{n^2+3}
				\end{equation*}
				The series diverges. Sometimes one test can give more information than
				others.
			\end{itemize}
	\end{itemize}
%%END_EXAMPLE
%%END_BLOCK-------------------------------------------------------------------%%

%%BEGIN_BLOCK-----------------------------------------------------------------%%
	\Proposition Root test, Let $\DS s_n = \sumn{x_k}$ a series, $\DS \alpha =
	\limsupn \sqrt[n]{\abs{x_n}}$:
	\begin{enumerate}[label=(\roman*)]
		\item $s_n$ converges absolutely if $\alpha < 1$
		\item $s_n$ diverges if $\alpha > 1$
	\end{enumerate}

%%BEGIN_PROOF
	\Proof $\alpha = \limsupn \sqrt[n]{\abs{x_n}}, \Ep > 0, \alpha + \Ep
	< 1$:
	\begin{equation*}
		\limsupn \sqrt[n]{\abs{x_n}} = \limn \sup\{\sqrt[k]{\abs{x_k}} : k > n\}
		\Rightarrow \exists N : \abs{\sup\{\sqrt[n]{\abs{x_n}} : n > N\} - \alpha} <
		\Ep
	\end{equation*}
	\begin{equation*}
		\alpha - \Ep < \abs{\sup\{\sqrt[n]{\abs{x_n}} : n > N\}} < \alpha + \Ep
		\Rightarrow \sqrt[n]{\abs{x_n}} < \alpha + \Ep \iff
		\abs{x_n} < (\alpha + \Ep)^n
	\end{equation*}
	Since the geometric series $\DS \series (\alpha + \Ep)^n < \infty$, then
	$\DS \series \abs{x_n} < \series (\alpha + \Ep)^n < \infty$, the series
	converges absolutely.
\\\qed
%%END_PROOF

%%BEGIN_EXAMPLE
	\Example
	\begin{itemize}
		\item $\DS \series\left({-\frac{1}{3}}\right)^n \Rightarrow
		\limsupn \sqrt[n]{-\frac{1}{3}^n} = \limsupn \frac{1}{3} = \frac{1}{3} < 1$,
		the series converges absolutely
		\item $\DS \series 2^{(-1)^n-n} \Rightarrow \sqrt[n]{2^{(-1)^n-n}} =
		\begin{cases}
			2^{\frac{1}{n}-1}  & \text{if } n \text{ even} \\
			2^{-\frac{1}{n}-1} & \text{if } n \text{ odd}
		\end{cases} \Rightarrow
		\limn 2^{\frac{1}{n}-1} = \limn 2^{-\frac{1}{n}-1} = \frac{1}{2} < 1$,
		the series converges.
	\end{itemize}
%%END_EXAMPLE
%%END_BLOCK-------------------------------------------------------------------%%

\newpage
%------------------------------------------------------------------------------%
% FUNCTIONS AND CONTINUITY
%------------------------------------------------------------------------------%
\section{Functions and continuity}

%%BEGIN_BLOCK-----------------------------------------------------------------%%
	\Def Given a function $\func{f}{X}{Y}$, the image of $f$ is defined as:
	$Im_f(X) = \{ f(x) : x \in X \}$. It contains all the images of all elements
	of $X$.
%%END_BLOCK-------------------------------------------------------------------%%

%%BEGIN_BLOCK-----------------------------------------------------------------%%
	\Def Given a function $\func{f}{X}{Y}$, the preimage of $f$ is defined as:
	$PreIm_f(Y) = \{ x : f(x) \in Y \}$. It contains all the elements of $X$ that
	have an image in $Y$.
%%END_BLOCK-------------------------------------------------------------------%%

%%BEGIN_BLOCK-----------------------------------------------------------------%%
	\Def Continuity of $\func{f}{(X,d_x)}{(Y,d_y)}$ (in a metric space)
\\$f$ is continuous at $x \in X$ if:
	\begin{equation*}
		\forall \Ep > 0 \ \exists \delta_\Ep > 0 : \forall x' \in X,
		d_x(x,x') < \delta_\Ep \Rightarrow d_y(f(x),f(x')) < \Ep
	\end{equation*}

%%BEGIN_EXAMPLE
	\Example Let's take $\DS f(x) = \begin{cases} 0 & \text{if } x = 0 \\ x^2
	\sin\frac{1}{x} & \text{otherwise} \end{cases}$
\\We want to prove that $f$ is continuous in 0. Let $\Ep > 0$, then
	$\abs{f(x) - f(0)} = \abs{f(x) - 0} = \abs{f(x)} \leq x^2$. If we take $\delta
	= \sqrt{\Ep}$, then:
	\begin{equation*}
		\abs{x - 0} < \delta \Rightarrow x^2 < \delta \Rightarrow \abs{f(x) - f(0)}
		\leq x^2 < \delta^2 = \Ep \Rightarrow f \text{ is continuous in 0}
	\end{equation*}
%%END_EXAMPLE

	\Remark Continuity can also be defined as follows:
	\begin{equation*}
		\forall \Ep > 0 \ \exists \delta_\Ep > 0 :
		Im_f(B_{\delta\Ep}^{d_x}(x)) \subseteq B_\Ep^{d_y}(f(x))
	\end{equation*}
	This means that the image of each ball around each $x$ is contained in another
	ball around $f(x)$.
%%END_BLOCK-------------------------------------------------------------------%%

%%BEGIN_BLOCK-----------------------------------------------------------------%%
	\Def Continuity of $\func{f}{(X,\tau_x)}{(Y,\tau_y)}$ (in a topological space)
\\$f$ is continuous at $x \in X$ if for all $U \in \tau_y$ such that $f(x) \in
	U$, then $PreIm_f(U) \in \tau_x$.

%%BEGIN_EXAMPLE
	\Example Let's take $(M,\tau_m)$, $(N,\tau_n)$, $M = N = \{ 1, 2 \}$,
	$\tau_m = \{ \emptyset, \{ 1 \}, \{ 2 \}, \{ 1, 2 \} \}$, $\tau_m = \{
	\emptyset, \{ 1, 2 \} \}$.
	\begin{itemize}
		\item Let $\func{f}{(M,\tau_m)}{(N,\tau_n)}$, such that $f(1) = 2$ and $f(2)
		= 1$:
		\begin{equation*}
			PreIm_f(\emptyset) = \emptyset \in \tau_m,
			PreIm_f(\{ 1, 2 \}) = \{ 1, 2 \} \in \tau_m \Rightarrow
			f \text{ is continuous in all } x \in M
		\end{equation*}
		\item Let $\func{g}{(N,\tau_n)}{(M,\tau_m)}$, such that $f(1) = 2$ and $f(2)
		= 1$:
		\begin{equation*}
			PreIm_g(\{ 1 \}) = \{ 2 \} \notin \tau_n \Rightarrow
			g \text{ is not continuous}
		\end{equation*}
	\end{itemize}
%%END_EXAMPLE
%%END_BLOCK-------------------------------------------------------------------%%

%%BEGIN_BLOCK-----------------------------------------------------------------%%
	\Proposition Continuous functions map open sets into open sets. If
	$\func{f}{(X,d_x)}{(Y,d_y)}$ continuous, then $PreIm_f(A)$ is open, for all
	$A \subseteq Y$ open.

%%BEGIN_PROOF
	\Proof Let $A \subseteq Y$ open, $x \in PreIm_f(A)$, $f(x) \in A$. Then, it
	exists $\Ep > 0$ such that $B_\Ep^{d_y}(f(x)) \subseteq A$. Since
	$f$ is continuous, then it exists $\delta_\Ep$ such that:
	\begin{equation*}
		PreIm_f(B_{\delta_\Ep}^{d_x}(x)) \subseteq B_\Ep^{d_y}(f(x))
		\subseteq A \Rightarrow B_{\delta_\Ep}^{d_x}(x) \subseteq PreIm_f(A)
		\Rightarrow A \text{ is open}
	\end{equation*}
	\qed
%%END_PROOF
%%END_BLOCK-------------------------------------------------------------------%%

%%BEGIN_BLOCK-----------------------------------------------------------------%%
	\Theorem Continuous functions map limits to limits:
	\begin{equation*}
		f \text{ continuous, } x_n \to x \iff f(x_n) \to f(x)
	\end{equation*}

%%BEGIN_PROOF
	\Proof Topological (only for ``$\Rightarrow$'')
\\Let $\func{f}{(X,\tau_x)}{(Y,\tau_y)}$, $A \in \tau_y$, $f(x) \in A$. Since
	$f$ continuous, then $PreIm_f(A) \in \tau_x$ and $x \in PreIm_f(A)$. Since
	$x_n$ converges to $x$, we have that:
	\begin{equation*}
		\exists N : \forall n \geq N, (x_n) \subseteq PreIm_f(A) \Rightarrow
		Im_f(x_n) \subseteq A \Rightarrow f(x_n) \to f(x)
	\end{equation*}
	\qed
%%END_PROOF

%%BEGIN_PROOF
	\Proof Metrical (only for ``$\Rightarrow$'')
\\Let $\Ep > 0$, $\func{f}{(X,d_x)}{(Y,d_y)}$ continuous. Then, it exists
	$\delta > 0$ such that for all $x' \in X$, $d_x(x,x') < \delta$. This means
	$d_y(f(x),f(x')) < \Ep$. Since $x_n$ converges to $x$:
	\begin{equation*}
		\exists N : \forall n \geq N, d_x(x,x) < \delta \Rightarrow d_y(f(x_n),f(x))
		< \Ep \Rightarrow f(x_n) \to f(x)
	\end{equation*}
	\qed
%%END_PROOF

%%BEGIN_EXAMPLE
	\Example Let's take $f(x) = 2x^2 + 1$ and $\DS \limn x_n = x$. Then:
	\begin{equation*}
		\limn 2x_n^2 + 1 = 2 \left(\limn x_n\right)^2 + 1 = 2x^2 + 1
	\end{equation*}
	This means that for $x_n \to x$, then $f(x_n) \to f(x)$. Therefore, $f$ is
	continuous.
%%END_EXAMPLE
%%END_BLOCK-------------------------------------------------------------------%%

%%BEGIN_BLOCK-----------------------------------------------------------------%%
	\Proposition $\funcR{f,g}$ continuous at $x \Rightarrow f+g$, $f \cdot g$ and
	$\frac{f}{g}$ (for $g(x) \neq 0$) are continuous at $x$.
%%END_BLOCK-------------------------------------------------------------------%%

%%BEGIN_BLOCK-----------------------------------------------------------------%%
	\Proposition $f$ continuous at $x$ and $g$ continuous at $f(x) \Rightarrow
	g \circ f = g(f(x))$ is continuous at $x$.

%%BEGIN_PROOF
	\Proof
	\begin{enumerate}[label=(\arabic*)]
		\item $f$ continuous at $x \Rightarrow$ for $x_n \to x$, then $f(x_n) \to f(x)$
		\item $g$ continuous at $y \Rightarrow$ for $y_n \to y$, then $g(y_n) \to f(y)$
		\item In particular, for $y_n = f(x_n) \Rightarrow g(f(x_n)) \to g(f(x))$
	\end{enumerate}
	\qed
%%END_PROOF
%%END_BLOCK-------------------------------------------------------------------%%

%%BEGIN_BLOCK-----------------------------------------------------------------%%
	\Def $\func{f}{(X,d)}{(X,d)}$ is a contraction $\iff$ it exists $0 \leq c < 1$
	such that $d(f(x),f(y)) \leq cd(x,y)$, for all $x,y \in X$.
%%END_BLOCK-------------------------------------------------------------------%%

%%BEGIN_BLOCK-----------------------------------------------------------------%%
	\Theorem Banach fixed point
\\Let's take $(X,d)$ complete (Cauchy $\iff$ convergence) and
	$\func{f}{(X,d)}{(X,d)}$ a contraction, then:
	\begin{enumerate}[label=(\roman*)]
		\item $\exists! x^* \in X : f(x^*) = x^*$
		\item $x_0 \in X$, $x_{n+1} = f(x_n) \Rightarrow x_n \to x^*$
	\end{enumerate}
%%END_BLOCK-------------------------------------------------------------------%%


%------------------------------------------------------------------------------%
% LIMITS OF FUNCTIONS
%------------------------------------------------------------------------------%
\section{Limits of functions}

%%BEGIN_BLOCK-----------------------------------------------------------------%%
	\Def $f$ converges to $c$ at $x_0 \iff$ for all $(x_n)$ such that $x_n \to x_0$
	we have $f(x_n) \to c$. We write $\DS \limx{x_0} f(x) = c$.
	\begin{itemize}
		\item $f$ converges from above if, for all $(x_n)$, then $x_0 < x_n$.
		We write $\DS \limx{x_0^+} f(x) = c$.
		\item $f$ converges from below if, for all $(x_n)$, then $x_n < x_0$.
		We write $\DS \limx{x_0^-} f(x) = c$.
	\end{itemize}

%%BEGIN_EXAMPLE
	\Example
	\begin{itemize}
		\item Let $\DS f(x) = \frac{1}{x} \Rightarrow \limx{0^+} \frac{1}{x} =
		+\infty, \limx{0^-} \frac{1}{x} = -\infty$
		\item Let $\DS f(x) = floor(x) \Rightarrow \limx{1^+} floor(x) = 1,
		\limx{1^-} floor(x) = 0$, but $\DS \limx{\frac{1}{2}^+} floor(x) =
		\frac{1}{2} = \limx{\frac{1}{2}^-} floor(x)$
	\end{itemize}
%%END_EXAMPLE
%%END_BLOCK-------------------------------------------------------------------%%

%%BEGIN_BLOCK-----------------------------------------------------------------%%
	\Def $\funcR{f}$ is bounded on $X \subseteq \R$ if $Im(X) = \{ f(x) : x \in X
	\}$ is bounded. That is, it exists $c$ such that $\abs{f(x)} \leq c$ for all
	$x \in X$.

%%BEGIN_EXAMPLE
	\Example $\func{f}{\R}{\intcc{-1,1}}$, $f(x) = \sin(x)$ is bounded on $\R$,
	since $\abs{\sin(x)} \leq 1$ for all $x \in \R$.
%%END_EXAMPLE
%%END_BLOCK-------------------------------------------------------------------%%

%%BEGIN_BLOCK-----------------------------------------------------------------%%
	\Theorem Extreme value
\\If $\functoR{f}{\intcc{a,b}}$ is continuous, then:
	\begin{enumerate}[label=(\roman*)]
		\item $f$ is bounded on $\intcc{a,b}$
		\item $f$ has a maximum and a minimum on $\intcc{a,b}$
	\\$\exists x_{minimizer}, x_{maximizer} \in \intcc{a,b} :
	f(x_{mininizer}) \leq f(x) \leq f(x_{maximizer}), \forall x \in \intcc{a,b}$
	\end{enumerate}

%%BEGIN_PROOF
	\Proof
	\begin{enumerate}[label=(\roman*)]
		\item Proof by contradiction, we assume $f$ unbounded
	\\This means, for all $n \in \N$ it exists $x_n \in \intcc{a,b}$ such that
		$\abs{f(x_n)} > n$. Then, $(x_n) \subseteq \intcc{a,b}$ is bounded and has
		a subsequence $(x_{n_k})$ that converges to a $x_0 \in \intcc{a,b}$
		(Bolzano-Weierstrass). Since $f$ is continuous at $x_0$, then $f(x_{n_k})$
		converges to $f(x_0)$. If $f$ is unbounded, then $f(x_n)$ diverges:
		contradiction. This means $f$ is bounded.

		\item Let's take $M = \sup\{ f(x) : x \in \intcc{a,b} \}$ the smallest upper
		bound of $Im(\intcc{a,b})$, then $M - \frac{1}{n}$ is not an upper bound.
		We know it exists $x_n$ such that $M - \frac{1}{n} < f(x_n) \leq M$. This
		means:
		\begin{equation*}
			\limn{M - \frac{1}{n}} \leq \limn{f(x_n)} \leq M \iff
			M \leq \limn{f(x_n)} \leq M \iff \limn{f(x_n)} = M
		\end{equation*}
		Moreover, $(x_n) \subseteq \intcc{a,b}$ is bounded, and it has a subsequence
		$(x_{n_k})$ convergent to $x_0 \in \intcc{a,b}$. Since $f$ is continuous,
		then $f(x_{n_k})$ converges to $f(x_0)$. This means $f(x_0) = M$, where
		$x_0$ is the maximizer.
	\end{enumerate}
	\qed
%%END_PROOF

%%BEGIN_REMARK
	\Remark This isn't true if the interval is open:
	\begin{itemize}
		\item $\funcR{f}{\intoo{0,1}}$, $f(x) = \frac{1}{x}$ is unbounded, since
		$f(x)$ goes to infinity for $x$ small

		\item $\func{f}{\intoo{-1,1}}$, $f(x) = x^2$, doesn't have a max, since
		$\sup\{ Im(\intoo{-1,1}) \} = 1$ is $f(1)$ or $f(-1)$, but 1 and -1 $\notin
		\intoo{-1,1}$
	\end{itemize}
%%END_REMARK
%%END_BLOCK-------------------------------------------------------------------%%

%%BEGIN_BLOCK-----------------------------------------------------------------%%
	\Theorem Intermediate value (IVT)
\\$f$ continuous on $\intcc{a,b}$, $f(a) < c < f(b) \Rightarrow \exists x \in
	\intcc{a,b} : f(x) = c$.

%%BEGIN_PROOF
	\Proof Let's assume $f(a) < c < f(b)$ (the same can be done for the opposite).
	Let's have $S = \{ x \in \intcc{a,b} : f(x) < c \}$ not empty, since at least
	$f(a) \in S$. Let $x_0 = \sup S \in \intcc{a,b}$, then $x_0 - \frac{1}{n}$ is
	not an upper bound, and it exists $s_n \in S$ such that $x_0 - \frac{1}{n} <
	s_n \leq x_0$. This means $s_n$ converges to $x_0$. We now have $f(s_n) < c$
	and $f(x_0) = \lim f(s_n) \leq c$.
\\Let's take $t_n = \min \{ x_0 + \frac{1}{n}, b \} \in \intcc{a,b}$, where
	$x_0 < t_n \leq t_n + \frac{1}{n}$, meaning that $t_n$ converges to $x_0$. Now
	$t_n \notin S$ (since $t_n > \sup S$), $f(t_n) \geq c$ and $f(x_0) = \lim t_n
	\geq c$. Therefore $c \leq f(x_0) \leq c$, so $f(x_0) = c$.
\\\qed
%%END_PROOF
%%END_BLOCK-------------------------------------------------------------------%%

%%BEGIN_BLOCK-----------------------------------------------------------------%%
	\Def A Darboux function is a function that satisfies the intermediate value
	property.
%%END_BLOCK-------------------------------------------------------------------%%

%%BEGIN_BLOCK-----------------------------------------------------------------%%
	\Proposition Continuous implies Darboux, but not the opposite.

	%%BEGIN_PROOF
		%???
	%%END_PROOF

%%BEGIN_EXAMPLE
	\Example $f(x) = \begin{cases}\sin(\frac{1}{x}) & x > 0 \\ 0 & x = 0\end{cases}
	\quad$ is a Darboux function, but it is not continuous.
%%END_EXAMPLE
%%END_BLOCK-------------------------------------------------------------------%%

%%BEGIN_BLOCK-----------------------------------------------------------------%%
	\Proposition Continuous functions map intervals to intervals.

%%BEGIN_PROOF
	%\Proof TODO: notes are not clear, ask.
%%END_PROOF
%%END_BLOCK-------------------------------------------------------------------%%

%%BEGIN_BLOCK-----------------------------------------------------------------%%
	\Def Connectedness
\\Let $(X, \tau)$ a topological space, the $A \subseteq X$ is disconnected if
	the two equivalent definitions hold:
	\begin{itemize}
		\item There exist $U, V \in \tau$ such that:
		\begin{itemize}
			\item $(A \cap U) \cap (A \cap V) = \emptyset$, and
			\item $(A \cap U) \cup (A \cap V) = A$, and
			\item $A \cap U \neq \emptyset \neq A \cap V$
		\end{itemize}

		\item There exist $U, V \subseteq A$ such that:
		\begin{itemize}
			\item $A = U \cup V$, and
			\item $\overline{U} \cap V = \emptyset = U \cap \overline{V}$ ! NOT SURE !
		\end{itemize}
	\end{itemize}
	A set is connected if it is not disconnected.
%%END_BLOCK-------------------------------------------------------------------%%

%%BEGIN_BLOCK-----------------------------------------------------------------%%
	\Proposition Continuous functions preserve connectedness.
\\$\func{f}{(X,\tau_x)}{(Y,\tau_y)}$, $A \subseteq X$ connected in $(X,\tau_x)$,
	then $Im(A) \subseteq Y$ is connected in $(Y,\tau_y)$.

%%BEGIN_PROOF
	\Proof By contradiction. We suppose $A$ connected and $Im(A)$ disconnected.
\\Since $Im(A)$ is disconnected, exist $V_1, V_2 \in \tau_y$ such that:
	\begin{itemize}
		\item $(Im(A) \cap V_1) \cap (Im(A) \cap V_2) = \emptyset$, and
		\item $(Im(A) \cap V_1) \cup (Im(A) \cap V_2) = Im(A)$, and
		\item $Im(A) \cap V_1 \neq \emptyset \neq Im(A) \cap V_2$
	\end{itemize}
	Let $U_1 = PreIm(V_1)$ and $U_2 = PreIm(V_2)$ it follows (it should be proved)
	that:
	\begin{itemize}
		\item $(PreIm(A) \cap U_1) \cap (PreIm(A) \cap U_2) = \emptyset$, and
		\item $(PreIm(A) \cap U_1) \cup (PreIm(A) \cap U_2) = PreIm(A)$, and
		\item $PreIm(A) \cap U_1 \neq \emptyset \neq PreIm(A) \cap U_2$
	\end{itemize}
	This implies that $A$ is disconnected, contradiction. Therefore $Im(A)$ is
	connected.
\\\qed
%%END_PROOF
%%END_BLOCK-------------------------------------------------------------------%%

%%BEGIN_BLOCK-----------------------------------------------------------------%%
	\Def Uniform continuity
\\$\func{f}{(X,d_x)}{(Y,d_y)}$ is uniformely continuous on $X$ if:
\begin{equation*}
	\forall \Ep > 0 \quad \exists \delta_\Ep > 0 : \forall x,x' \in X : d_x(x,x')
	< \delta \Rightarrow d_y(f(x),f(x')) < \Ep
\end{equation*}

%%BEGIN_EXAMPLE
	\Example $f(x) = \frac{1}{x^2}$ in $\left[a, +\infty\right)$, $a > 1$. To show
	that $f$ is uniformely continuous, we have to show that for all $\Ep > 0$
	exists $\delta_\Ep > 0$ such that for all $x,y$ such that $\abs{x-y} < \delta$
	then $\abs{f(x)-f(y)} < \Ep$.
\\Let $\Ep > 0$ and $f(x) - f(y) = \frac{1}{x^2} - \frac{1}{y^2} =
	\frac{(x+y)(x-y)}{x^2 y^2}$. Then, since $a \leq x,y \ \forall x,y$:
\begin{equation*}
	\frac{(x+y)}{x^2 y^2} = \frac{x}{x^2 y^2} + \frac{y}{x^2 y^2} \leq \frac{2}{a^3}
\end{equation*}
	We chose $\delta = \frac{\Ep a^3}{2}$, then:
\begin{equation*}
	\forall x,y \geq a : \abs{x-y} < \delta \Rightarrow \abs{f(x)-f(y)} =
	\abs{x-y} \abs{\frac{x+y}{x^2 y^2}} < \delta \frac{2}{a^3} =
	\frac{\Ep a^3}{2} \frac{2}{a^3} = \Ep
\end{equation*}
	This mean $f$ is uniformely continuous.
%%END_EXAMPLE

	\Remark Uniform continuity is different from normal continuity. In normal
	continuity the $\delta$ depends on both $\Ep$ and $x$, while in uniform
	continuity $\delta$ depends solely on $\Ep$. In fact, $f$ is ``normally''
	continuous on $x_0 \in X$ if:
\begin{equation*}
	\forall \Ep > 0 \quad \exists \delta_{\Ep,x_0} > 0 : \forall x \in X :
	d_x(x_0,x) < \delta \Rightarrow d_y(f(x_0),f(x)) < \Ep
\end{equation*}
%%END_BLOCK-------------------------------------------------------------------%%

%%BEGIN_BLOCK-----------------------------------------------------------------%%
	\Theorem $f$ continuous on $A$, closed and bounded $\Rightarrow$ $f$ is
	uniformely continuous on $A$.
%%END_BLOCK-------------------------------------------------------------------%%

%%BEGIN_BLOCK-----------------------------------------------------------------%%
	\Theorem $f$ uniformely continuous on $S$, $(s_n) \subseteq S$ is Cauchy
	sequence $\Rightarrow f(s_n)$ is Cauchy sequence.

%%BEGIN_PROOF
	\Proof Let $(s_n) \subseteq S$ a Cauchy sequence, $\Ep > 0$ and $f$ uniformely
	continuous:
\begin{enumerate}
	\item Exists $\delta > 0$ such that $\abs{f(x)-f(y)} < \Ep$ for all $\abs{x-y}
	< \delta$.
	\item Exists $N_\Ep$ such that for all $n,m \geq N$, then $\abs{s_n - s_m} <
	\delta$
\end{enumerate}
	Combining (1) and (2) we have that for all $n,m \geq N$, $\abs{f(s_n)-f(s_m)}
	< \Ep$. This means $f(s_n)$ is a Cauchy sequence.
\\\qed
%%END_PROOF

%%BEGIN_EXAMPLE
	\Example $f(x) = \frac{1}{x^2}$ is not uniformely continuous on $(0,1)$. In
	fact, $s_n = \frac{1}{n}$ is Cauchy, but $f(s_n) = n^2$ is not Cauchy.
%%END_EXAMPLE
%%END_BLOCK-------------------------------------------------------------------%%

%%BEGIN_BLOCK-----------------------------------------------------------------%%
	\Def Sequence of functions
\\$(f_n) \subseteq \{ \func{f}{S}{\R} \}$ is a sequence of functions. A sequence
	of function can converge to a function: $f_n \to f$.

%%BEGIN_EXAMPLE
	\Example $f_n(x) = \frac{x}{n} \to f(x) = 0$
%%END_EXAMPLE
%%END_BLOCK-------------------------------------------------------------------%%

%%BEGIN_BLOCK-----------------------------------------------------------------%%
	\Def $f_n$ converges pointwise to $f \iff \DS \limn{f_n(x)} = f(x)$ for all
	$x \in S$.
\begin{equation*}
	\forall \Ep > 0, x \in S \exists N_\Ep : \abs{f_n(x) - f(x)} < \Ep
\end{equation*}

%%BEGIN_EXAMPLE
	\Example $f_n(x) = x^n$, $x \in \intcc{0,1} \Rightarrow f(x) =
	\begin{cases}1 & x = 1 \\ 0 & x \neq 1\end{cases}$. $f_n$ is continuous and
	$f$ is discontinuous.
%%END_EXAMPLE
%%END_BLOCK-------------------------------------------------------------------%%

%%BEGIN_BLOCK-----------------------------------------------------------------%%
	\Def $d_\infty(f_n,f) = \sup \{ \abs{f_n(x) - f(x)} < \Ep \}$
%%END_BLOCK-------------------------------------------------------------------%%

%%BEGIN_BLOCK-----------------------------------------------------------------%%
	\Def $f_n$ converges uniformely to $f$ if exists $N_\Ep$ such that
	$d_\infty(f_n,f) < \Ep$ for all $n \geq N_\Ep$.

%%BEGIN_EXAMPLE
	\Example Let $f_n(x) = (1 - \abs{x})^n$, $x \in \intoo{-1,1}$. Then $f$
	converges pointwise (but not uniformely) to $f(x) =
	\begin{cases}1 & x = 0 \\ 0 & x \neq 0\end{cases}$. In fact:
\begin{itemize}
	\item Pointwise convergence
\\For $x = 0$, $f_n(x) = (1 - 0)^n = 1$, then $\DS \limn{f_n} = \limn{1} = 1$.
\\For $x \neq 0$, $\abs{x} < 1$. This means $1 - |x| < 1$, then
	$\DS \limn{(1-|x|)^n} = 0$.

	\item Uniform convergence
\\We assume $f_n \tounif f$ and we take $\Ep = \frac{1}{2}$. Then it exists $N$
	such that $\abs{f_n(x) - f(x)} < \frac{1}{2}$ for all $x \in \intoo{-1,1}$.
\\Let's take $x = 1 - 2^{-\frac{1}{n}}$, then $1 - x = 2^{-\frac{1}{n}}$. Thus
	$(1 - x)^n = (2^{-\frac{1}{n}})^n = \frac{1}{2} \nless \frac{1}{2} = \Ep$.
	Contradiction, $f$ doesn't converge uniformely to $f$.
\end{itemize}
%%END_EXAMPLE
%%END_BLOCK-------------------------------------------------------------------%%

%%BEGIN_BLOCK-----------------------------------------------------------------%%
	\Theorem Uniform limit of a continuous function is continuous
\\$f_n(x)$ continuous and $f_n(x) \tounif f(x) \Rightarrow f(x)$ is continuous.

%%BEGIN_PROOF
	\Proof Let $\Ep > 0$
\\Since $f_n \tounif f(x)$, it exists $N_\Ep$ such that $\abs{f_n(x)-f(x)} <
	\frac{\Ep}{3}$ for all $n \geq N$.
\\Since $f_n$ continuous, it exists $\delta > 0$ such that for all $x,x_0$ such
	that $\abs{x_0 - x} < \delta$, then $\abs{f_N(x_0) - f_N(x)} < \frac{\Ep}{3}$.
\\By triangle inequality we have:
\begin{equation*}
	\abs{f(x_0) - f(x)} \leq \abs{f(x_0) - f_N(x_0)} + \abs{f_N(x_0) - f(x)} \leq
\end{equation*}
\begin{equation*} \leq
	\abs{f(x_0) - f_N(x_0)} + \abs{f_N(x_0) - f_N(x)} + \abs{f_N(x) - f(x)} <
	\frac{\Ep}{3} + \frac{\Ep}{3} + \frac{\Ep}{3} = \Ep
\end{equation*}
	\qed
%%END_PROOF

%%BEGIN_EXAMPLE
	\Example
\begin{itemize}
	\item %1
	Let $f_n \tounif f$ and $g_n \tounif g$ on $S \subseteq \R$. Then $f_n + g_n
	\tounif f + g$. In fact:
\begin{equation*}
	\exists N_f : \forall x \in S \abs{f_n(x) - f(x)} < \frac{\Ep}{2} \forall n > N_f
\end{equation*}
\begin{equation*}
	\exists N_g : \forall x \in S \abs{g_n(x) - g(x)} < \frac{\Ep}{2} \forall n > N_g
\end{equation*}
	We take $N = \max \{ N_f, N_g \}$. Then:
\begin{equation*}
	\abs{f_n(x) - f(x) + g_n(x) - g(x)} \leq \abs{f_n(x) - f(x)} + \abs{g_n(x) -
	g(x)} < \frac{\Ep}{2} + \frac{\Ep}{2} = \Ep, \ \forall n \geq N
\end{equation*}
	This means $f_n + g_n \tounif f + g$.

	\item %2
	Let $f_n \tounif f$ and $g_n \tounif g$ on $S \subseteq \R$. Then $f_n g_n$
	doesn't converge uniformely to $fg$. In fact, let $h_n(x) = \frac{x}{n}$.
	By contradiction we can prove $h_n$ doesn't converge uniformely to $h$. Now,
	if we take $f_n(x) = \frac{1}{n}$ and $g_n(x) = x$ (uniformely convergent),
	then $f(x) g(x) = \frac{x}{n} = h(x)$ not uniformely convergent. We found a
	counter example.

	\item %3
	Let $f_n(x)$ continuous on $\intcc{a,b}$, $f_n(x) \tounif f(x)$, $(x_n)
	\subseteq \intcc{a,b}$ and $x_n \to x$. Then, $f_n(x_n) \to f(x)$. To prove it
	we have to show that exists $N$ such that for all $n \geq N$, then
	$\abs{f_n(x_n) - f(x)} < \Ep$.
	\begin{enumerate}
		\item $f_n \tounif f$, this means it exists $N_1$ such that $\abs{f_n(y) -
		f(y)} < \frac{\Ep}{2}$, for all $n \geq N_1$ and $y \in \intcc{a,b}$. In
		particular, $\abs{f_n(x_n) - f(x_n)} < \frac{\Ep}{2}$.

		\item Since $f_n(x)$ continuous and $f_n(x) \tounif f(x)$, then $f(x)$ is
		continuous. Then $f(x_n) \to f(x)$, this means it exists $N_2$ such that
		for all $n \geq N_2$, then $\abs{f(x_n) - f(x)} < \frac{\Ep}{2}$.

		\item We chose $N = \max \{ N_1. N_2 \}$. Then:
		\begin{equation*}
			\abs{f_n(x_n) - f(x)} \leq \abs{f_n(x_n) - f(x_n)} + \abs{f(x_n) - f(x)} <
			\frac{\Ep}{2} + \frac{\Ep}{2} = \Ep, \ \forall n \geq N
		\end{equation*}
	\end{enumerate}
	We can conclude that $f_n(x_n) \to f(x)$.
\end{itemize}
%%END_EXAMPLE
%%END_BLOCK-------------------------------------------------------------------%%


%------------------------------------------------------------------------------%
% POWER SERIES
%------------------------------------------------------------------------------%
\section{Power series}

%%BEGIN_BLOCK-----------------------------------------------------------------%%
	\Def Power series
\\Let $(a_n)_{n \geq 0} \subseteq \R$ a sequence. Then $\pseries{a_n x^n}$ is a
	power series. We have three cases:
	\begin{itemize}
		\item The series converges for all $x \in \R$.
		\item The series converges for $x = 0$ only.
		\item The series converges for some bounded interval.
	\end{itemize}
%%END_BLOCK-------------------------------------------------------------------%%

%%BEGIN_BLOCK-----------------------------------------------------------------%%
	\Theorem Let $\beta = \limsup \sqrt[n]{|a_n|}$ and $R = \frac{1}{\beta}$ ($R =
	\infty$ if $\beta = 0$, $R = 0$ if $\beta = \infty$). Then $\DS
	\pseries{a_n x^n}$:
	\begin{itemize}
		\item Converges for $|x| < R$.
		\item Diverges for $|x| > R$.
	\end{itemize}
	The same can be done with $\beta = \limsup \abs{\frac{a_n}{a_{n+1}}}$.

%%BEGIN_PROOF
	\Proof With root test. Let $\alpha = \limsup \sqrt[n]{|a_n|}$, then
	$\DS \series{a_n} < \infty$ if $\alpha < 1$ or $\DS \series{a_n} = \infty$ if
	$\alpha = 1$. Let $\alpha_x = \limsup \sqrt[n]{|a_n x^n|} =
	\limsup |x|\sqrt[n]{|a_n|} = |x| \limsup \sqrt[n]{|a_n|} = \beta |x|$. Then:
	\begin{enumerate}
		\item If $0 < R < \infty$, then $\alpha_x = \beta |x| = \frac{|x|}{R}$.
		\begin{itemize}
			\item If $|x| < R$, then $\alpha_x < 1$, by root test $\DS
			\pseries{a_n x^n}$ converges
			\item If $|x| > R$, then $\alpha_x > 1$, by root test $\DS
			\pseries{a_n x^n}$ diverges
		\end{itemize}

		\item If $R = \infty$, then $\alpha_x = 0 < 1$ independently of $x$. The
		series always converges.

		\item If $R =0$, then $\alpha_x = \infty > 1$ independently of $x$. The
		series always diverges.
	\end{enumerate}
	\qed
%%END_PROOF

%%BEGIN_EXAMPLE
	\Example
	\begin{itemize}
		\item %1
		Let $a_n = 1$. We have the power series $\DS \pseries{x^n}$ and $\beta =
		\limsup \sqrt[n]{1} = 1$, then $R = 1$. This means the series converges for
		$x \in \intoo{-1,1}$ and diverges for $x$ such that $|x| > 1$. Moreover, it
		diverges for $x = 1$, since $\DS \pseries{1} = +\infty$, and it is not
		defined for $x = -1$.

		\item %2
		Let $\DS \pseries{\frac{1}{n}x^n}$ a power series. Then $\limsup
		\abs{\frac{a_n}{a_{n+1}}} = \limsup \abs{\frac{n}{n+1}} = 1$, then $R = 1$.
		For $x = 1$ we have the harmonic series $\DS \pseries{\frac{1}{n}}$ wich
		diverges, for $x = -1$ we have $\DS \pseries{\frac{(-1)^n}{n}} < \infty$. We
		can conclude that the power series converges for $x \in \intco{-1,1}$.
	\end{itemize}
%%END_EXAMPLE
%%END_BLOCK-------------------------------------------------------------------%%

%------------------------------------------------------------------------------%
% END
%------------------------------------------------------------------------------%

\end{document}
